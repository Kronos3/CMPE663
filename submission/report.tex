% A skeleton file for producing Computer Engineering reports
% https://kgcoe-git.rit.edu/jgm6496/KGCOEReport_template

\documentclass[CMPE]{KGCOEReport}

% The following should be changed to represent your personal information
\newcommand{\classCode}{CMPE 663}  % 4 char code with number
\newcommand{\name}{Andrei Tumbar}
\newcommand{\LabSectionNum}{1}
\newcommand{\LabInstructor}{Wolfe}
\newcommand{\TAs}{Nitin Borhade}
\newcommand{\exerciseNumber}{G}
\newcommand{\exerciseDescription}{Ultrasound}

\usepackage{tikz}
\usepackage{circuitikz}
\usetikzlibrary{calc}
\usepackage{multirow}
\usepackage{titlesec}
\usepackage{float}
\usepackage{lmodern}
\usepackage{siunitx}
\usepackage{subcaption}
\usepackage{graphicx}
\usepackage[usestackEOL]{stackengine}
\usepackage{scalerel}
\usepackage[T1]{fontenc}
\usepackage{amsmath}


\def\code#1{\texttt{#1}}

\begin{document}
    \maketitle
    \section*{Analysis/Design}

    This project looked at operating an ultrasonic sensor for distance detection.
    Operation of the ultrasonic involves input capture for echo timing as well as
    polling for accurate pulse widths. \code{TIM2} was used to operate the timing
    of input and output to the sensor. The UART was used to take user as usual.
    While taking sensor measurements, UART input can halt the measurements if
    any input is detected. To implement, a non-blocking check for UART input can
    be embedded into the loop for taking sensor measurements.

	\subsection*{Ultrasonic sensor signals}

	Other than the voltage and ground wires, the ultrasonic sensor operates with two
	signals: trigger and echo. The trigger is an input to the sensor which tell the
	sensor when to fire its output pulse. The echo pin is an output from the sensor
	that will notify when the output ping is sent and when it returns.

	\begin{figure}[h!]
      \centering
      \includegraphics[width=5.5in]{Ultrasonic}
      \caption{Ultrasonic sensor inputs and outputs}
      \label{fig:us}
    \end{figure}

	Figure \ref{fig:us} shows the pin operation for the input and output to the
	sensor. The output trigger can be easily implementing by setting the trigger
	pin high, waiting for $\SI{\sim 20}{\micro\s}$ and then setting it back to low.
	The echo pin timing can be implemented by enabling input capture with both rising
	and falling triggers. The first event on the timer will provide a reference count
	to substract the second event from. The counter different will tell us how long the
	signal took to return. The speed of sound can be used to calculate the distance the
	sound wave travelled.

	\subsection*{Distance calculation}

	Calculating the distance between detected surface and the sensor is fairly
	straight forward. I used a prescaler of $79$ on \code{TIM2} with a system
	clock of \SI{80}{\mega\Hz}. This means that each count on the timer represents
	\SI{1}{\micro\s}. Using this informating, we can derive an expression.

	\begin{align*}
		c &= \SI{343}{\m/\s} && \text{Speed of sound in air}\\
		\SI[parse-numbers=false]{t_{response}}{\micro\s} &= t_{reply} - t_{reference} && \text{Elasped time of wave travel}\\
		\SI[parse-numbers=false]{d_{travel}}{\milli\m} &= \frac{\SI[parse-numbers=false]{t_{response}}{\micro\s}}{\SI{1000}{\micro\s/\milli\s}} \cdot \SI[parse-numbers=false]{c}{\m/\s} && \text{Total travel of distance in mm}\\
		d &= \frac{d_{travel}}{2} && \text{Travel is round trip distance}
	\end{align*}

	Using this conversion with a system clock and timer prescaler as stated above,
	the expression was verified to be correct by placing the sensor a known length
	away from a surface.

    \section*{Test Plan}

    Testing the program involved pointing the sensor at a wall and testing the readings
    at varying distances. I implemented a "live" feed before the 100 measurements are
    taken to make creating the CSV easier. The ultrasonic sensor outputed values that
    were consistent with visual inspection. This confirmed the functionality of the
    sensor was correct. Testing revealed that my equation to convert timer ticks to
    distance was incorrect. I was not taking into account the fact that the travel
    time is a \emph{round-trip} time. This means that you must divide this time by to
    calculate the distance the sound wave travelled.

    \section*{Project Results}

    When testing the sensor measurements on different surfaces, we can see a varying
    standard deviation. For softer surfaces like a pillow or clothing, the deviation
    was significantly higher than for smooth and hard surfaces like a wall. The accuracy
    of the sensor was fairly high for distances between \SI{100}{\milli\m} and \SI{3}{\m}.
    At distances that were out of range, the pulse width was capped off by the hardware.
    When measuring larger distances, $>\SI{1}{\m}$, a large surface is needed because the
    cone radius rises as the distance away from the sensor increases.

    \section*{Lessons Learned}

	This project was fairly straight forward. The code to implement this project was
	very similar to the first project. One difference between the two was the fact 
	that this one required a timing output signal. I implemented this in the simplest
	form possible, setting the pin high, polling the timer counter, setting the pin
	to low. Before this project I was not familiar with the operation of the ultrasonic
	sensor.

\end{document}
